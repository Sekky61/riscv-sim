
\chapter{Závěr}

V~této práci byly prozkoumány některé známé metody efektivní implementace superskalárních procesorů.
Detailněji byla probrána instrukční sada RISC-V, kterou simulátor využívá.
Také byla uvedena teorie implementace webových aplikací s~přihlédnutím k~návrhu uživatelských rozhraní.

Byla provedena důkladná analýza stávajícího simulátoru, zhodnocení jeho kladů i nedostatků.
Na základě analýzy byla navržena mnohá zlepšení, jak z~pohledu implementace simulace, tak použitelnosti jeho rozhraní.

Z~mého pohledu nejdůležitější změnou je samotné přenesení simulátoru na webovou platformu.
Simulátor je nyní všem dostupný přímo z~prohlížeče.

Do rozhraní simulátoru se podařilo přidat režim příkazové řádky a HTTP server.
Důležitou součástí simulátoru je integrace s~překladačem GCC a nový parser assembleru, který umožňuje simulovat širokou škálu programů v~jazyce C.
Společně s~množstvím před-připravených příkladů to uživatelům umožňuje rychle experimentovat se složitějšími programy, než jaké by mohli nebo chtěli psát ručně.
Simulátor také sbírá běhové statistiky a je možné definovat data, nad kterými má program pracovat.

Bylo také provedeno mnoho změn vnitřního fungování simulátoru.
Projekt obsahuje množství testů, dokumentace a skriptů k~instalaci, spuštění a nasazení do provozu.
Příloha \ref{prevzanaPrace} uvádí tabulku s~přehledem jednotlivých modulů aplikace a můj přínos.

Aplikace je v~současné podobě dobře provozuschopná, o~čemž svědčí kladný ohlas z~uživatelského dotazníku.
Na webové stránce je možné provozovat interaktivní simulaci, měnit konfiguraci, editovat kód a prohlížet statistiky o~simulaci.
Grafické zpracování a provázání prvků v~simulátoru přináší studentům přehledný zdroj k~výuce fungování moderních procesorů.

Přínos projektu pro výuku hardwaru vnímám velmi pozitivně.
Doufám, že aplikace najde své využití nejen na FIT VUT v~Brně, ale i v~široké studentské komunitě.

Budoucí práce se může zaměřit na další rozšíření instrukční sady, například na vektorové instrukce.
Dalším zajímavým směrem může být možnost programovatelného přerušení simulace v~určitém bodě (\emph{breakpoint}).
